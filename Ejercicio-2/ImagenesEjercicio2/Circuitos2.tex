\documentclass[border={0.5cm 0.5cm 0.5cm 0.5cm}, 11pt, tikz, multi=page]{standalone}
\usepackage[utf8]{inputenc}
\usepackage[spanish, es-tabla, es-noshorthands]{babel}

\usepackage[a4paper, footnotesep = 1cm, width=18cm, left=2cm, top=2.5cm, height=25cm, textwidth=18cm, textheight=25cm]{geometry}
%\geometry{showframe}

\usepackage{tikz}
\usepackage{textcomp}
\usetikzlibrary{shapes,arrows}

\usepackage{amsmath}
\usepackage{amsfonts}
\usepackage{amssymb}
\usepackage{float}
\usepackage{graphicx}
\usepackage{caption}
\usepackage{subcaption}
\usepackage{multicol}
\usepackage{multirow}
\setlength{\doublerulesep}{\arrayrulewidth}
\usepackage{booktabs}

\usepackage{hyperref}
\hypersetup{
    colorlinks=true,
    linkcolor=blue,
    filecolor=magenta,      
    urlcolor=blue,
    citecolor=blue,    
}

\newcommand{\quotes}[1]{``#1''}
\usepackage{array}
\newcolumntype{C}[1]{>{\centering\let\newline\\\arraybackslash\hspace{0pt}}m{#1}}
\usepackage[american]{circuitikz}
\usepackage{fancyhdr}
\usepackage{units}

% Definition of blocks:
\tikzset{%
  block/.style    = {draw, thick, rectangle, minimum height = 3em,
    minimum width = 3em},
  sum/.style      = {draw, circle, node distance = 2cm}, % Adder
  input/.style    = {coordinate}, % Input
  output/.style   = {coordinate}, % Output
  >=Stealth
}

% Defining string as labels of certain blocks.
\newcommand{\suma}{\Large $\Sigma$}
\newcommand{\inte}{$\displaystyle \int$}
\newcommand{\derv}{\huge $\frac{d}{dt}$}

\begin{document}

\begin{page}
\begin{circuitikz}

	%DIBUJO EL DFF
	\ctikzset{multipoles/thickness=3}
	\ctikzset{multipoles/dipchip/width=1.25}
	\draw (0,0) node[dipchip, num pins=8, hide numbers, no topmark, external pins width=0](C1){};
	\node [right, font=\footnotesize] at (C1.bpin 1) {D};
	\node [left, font=\footnotesize] at (C1.bpin 8) {$Q$};
	\draw (C1.bpin 4) ++(0,0.1) -- ++(0.1,-0.1) node[right, font=\footnotesize]{CLK} -- ++ (-0.1,-0.1);	
	\draw (C1.bpin 4) -- ++(-0.3,0) node[](CLK1){} -- ++(-0.5,0) node[circ, label=left:$CLK$](){};	
	%SALIDA
	\draw (C1.bpin 8) -- ++(0.3,0) node[](Q1){} -- ++(0.5,0) node[circ, label=right:$y_1$](){};	
	%ENTRADA	
	\draw (C1.bpin 1) -- ++(-0.3,0) node[](D1){} -- ++(-0.5,0) node[american or port, anchor = out](or1){};
	
	%DE ATRAS PARA ADELANTE, DE ABAJO PARA ARRIBA:
	%PRIMER OR Y PRIMER AND
	\draw (or1.in 2) -| ++(-0.5,-0.5) node[american and port, anchor = out, number inputs=3](and1){};
	\draw (and1.in 3) -- ++(-0.5,0) node[circ, label=left:$\overline{y_1}$](){};
	\draw (and1.in 2) -- ++(-0.5,0) node[circ, label=left:$\overline{y_2}$](){};
	\draw (and1.in 1) -- ++(-0.5,0) node[circ, label=left:$In$](){}; 	
	
	%SEGUNDO OR
	\draw (or1.in 1) -| ++(-0.5,1) node[american or port, anchor = out](or2){};
	
	%SEGUNDO AND
	\draw (or2.in 2) |- ++(-0.5,-0.5) node[american and port, anchor = out, number inputs=3](and2){};
	\draw (and2.in 3) -- ++(-0.5,0) node[circ, label=left:$\overline{y_1}$](){};
	\draw (and2.in 2) -- ++(-0.5,0) node[circ, label=left:$y_2$](){};
	\draw (and2.in 1) -- ++(-0.5,0) node[circ, label=left:$\overline{In}$](){};
	
	%TERCER AND
	\draw (or2.in 1) |- ++(-0.5,1) node[american and port, anchor = out, number inputs=3](and3){};
	\draw (and3.in 3) -- ++(-0.5,0) node[circ, label=left:$y_1$](){};
	\draw (and3.in 2) -- ++(-0.5,0) node[circ, label=left:$y_2$](){};
	\draw (and3.in 1) -- ++(-0.5,0) node[circ, label=left:$In$](){};
	
	\draw (and3.out) -- ++(1,0) node[circ, label=above:$Out$](){};

	%CIRCUITO DE ABAJO

	%DIBUJO EL DFF
	\ctikzset{multipoles/thickness=3}
	\ctikzset{multipoles/dipchip/width=1.25}
	\draw (0,-4) node[dipchip, num pins=8, hide numbers, no topmark, external pins width=0](C1){};
	\node [right, font=\footnotesize] at (C1.bpin 1) {D};
	\node [left, font=\footnotesize] at (C1.bpin 8) {$Q$};
	\draw (C1.bpin 4) ++(0,0.1) -- ++(0.1,-0.1) node[right, font=\footnotesize]{CLK} -- ++ (-0.1,-0.1);	
	\draw (C1.bpin 4) -- ++(-0.3,0) node[](CLK1){} -- ++(-0.5,0) node[circ, label=left:$CLK$](){};	
	%SALIDA
	\draw (C1.bpin 8) -- ++(0.3,0) node[](Q1){} -- ++(0.5,0) node[circ, label=right:$y_2$](){};	
	%ENTRADA	
	\draw (C1.bpin 1) -- ++(-0.3,0) node[](D1){} -- ++(-0.5,0) node[american or port, anchor = out](or1){};
	
	%DE ATRAS PARA ADELANTE, DE ABAJO PARA ARRIBA:
	%PRIMER OR Y PRIMER AND
	\draw (or1.in 2) -| ++(-0.5,-0.5) node[american and port, anchor = out, number inputs=3](and1){};
	\draw (and1.in 3) -- ++(-0.5,0) node[circ, label=left:$y_1$](){};
	\draw (and1.in 2) -- ++(-0.5,0) node[circ, label=left:$\overline{y_2}$](){};
	\draw (and1.in 1) -- ++(-0.5,0) node[circ, label=left:$In$](){}; 	
	
	%SEGUNDO AND
	\draw (or1.in 1) -| ++(-0.5,0.5) node[american and port, anchor = out, number inputs=3](and2){};
	\draw (and2.in 3) -- ++(-0.5,0) node[circ, label=left:$\overline{y_1}$](){};
	\draw (and2.in 2) -- ++(-0.5,0) node[circ, label=left:$y_2$](){};
	\draw (and2.in 1) -- ++(-0.5,0) node[circ, label=left:$\overline{In}$](){};
\end{circuitikz}
\end{page}

\end{document}